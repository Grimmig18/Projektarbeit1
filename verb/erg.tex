\chapter{Ergebnisse und Bewertung der verschiedenen Algorithmen}\label{sec:result}
Nachdem nun drei Algorithmen zur Generierung einer Route und zwei zu deren Nachbesserung vorgestellt wurden, gilt es nun Erkenntnisse über die Ergebnisse dieser Algorithem im Vergleich zueinander zu gewinnen.
Dafür wurde ein Testszenario angelegt, in dem verschiedene Kombinationen dieser Algorithmen mehrmals getestet werden.
Jeder der drei Algorithmen zur Routengenerierung wird für sich, mit Auflösen von Überkreuzungen, mit Nachbesserung und mit Auflösen von Überkreuzungen und Nachbesserung getestet, wodurch es zwölf verschiedene Kombinationen gibt.
Diese Kombinationen werden $1\,000$ mal mit je $10,15,\cdots,100$ Knoten getestet und die Ergebnisse (Gesamtdistanz des finalen Graphs) vermerkt, wodurch $19\,000$ Datensätze entstehen. 
Wie aussagekräftigt diese Ergebnisse sind wird in dieser Arbeit nicht diskutiert.
\\
Tabelle %TODO: \vref{tab:test-results1} 
zeigt die möglichen Kombinationen der vorgestellten Algorithmen, wie häufig eine Kombination die beste Lösung generiert hat und die Anteil der Kombination an den $19\,000$ besten Ergebnissen.
Eindeutig kann hier erkannt werden, dass -- unabhängig von der Strategie -- Routen, die nach Algorithmus \vref{alg:after-control} nachgebessert werden, den größten Teil der besten Lösung beanspruchen.
Insgesamt erzielten die Kombinationen, die nachträglich den Nachbesserungs-Algorithmus anwenden $13\,083$ von $19\,000$ (68,86\%) besten Ergebnissen.
Den größten Teil macht dabei die Kombination, die den Insert-First-Algorithmus verwendet, aus.
Danach folgen ebenfalls nur mit Nachbesserung der Insert-Furthest-Algorithmus mit $4\,228$ und der Insert-Closest-Algorithmus mit $3\,648$ besten Ergebnissen.
Insgesamt verwenden 11,69\% der besten Lösungen nur einen Algorithmus zu Generierung des Graphen, 0,72\% lösen zusätzlich Überkreuzungen auf, 68,86\% verwenden den Algorithmus \vref{alg:after-control} zur Nachbesserung und 18,73\% beide dieser Verfahren.
Von den $19\,000$ besten Ergebnissen entfallen $7\,519$ (39,57\%) an eine Kombination mit dem Insert-First-Algorithmus, $5\,279$ (27,78\%) an den Insert-Closest-Algorithmus und $6\,202$ (32,64\%) an den Insert-Furthest-Algorithmus.
Weiterhin lässt sich aus den Datensätzen ablesen, dass die Differenz zwischen den Ergebnissen der besten und der schlechtesten Kombination mit zunehmender Knotenanzahl steigt.
Weicht die schlechteste Lösung bei 15 Knoten nur um 2,75\% von der Besten ab sind es bei 100 Knoten bereits 8,23\%.
\\
Weiterhin kann festgestellt werden, dass Kombinationen, die sich bei der Nachbesserung nur auf das Entfernen von Überkreuzungen verlassen, einen sehr geringen Teil der besten Ergebnisse ausmachen.
Dies kann allerdings damit erklärt werden, dass wenn ein generierter Graph von Beginn an keine Überkreuzungen enthält, die Kombination ohne Ausbesserung dieser Überkreuzungen die gleiche Gesamtdistanz vorweist wie eine Kombination, die Überkreuzungen entfernt.
Ist dies der Fall wird nur erste Kombination gezählt.
\\\\
Als statistisch beste Kombination kann also der Insert-First-Algorithmus in Verbindung mit der in \vref{sec:after-control} beschriebenen Nachbesserung angesehen werden.
Dies ist entgegen der gestellten Erwartungen, da in \vref{sec:insert-closest-verfahren} und \vref{sec:insert-furthest-verfahren} versucht wurde den Insert-First-Algorithmus zu verbessern.
Nach Auswertung der Daten scheint es aber so, als seien die Ergebnisse der Variationen des Insert-First-Verfahrens insgesamt schlechter als die des Originals.
Vergleicht man die durchschnittlichen Ergebnisse des Insert-First-Algorithmus in Kombination mit dem Nachbesserungs-Algorithmus mit den besten Durchschnittswerten aller Kombinationen, kann festgestellt werden, dass diese Kombination durchschnittlich 0,49\% vom besten Durchschnitt abweicht.



\begin{center}
    \begin{table}
        \begin{tabular}{ l | l | l | l | l }
            \textbf{Strategie} & \textbf{Überkreuzungen} & \textbf{Nachbesserung} & \textbf{\# Beste} & \textbf{\% Beste} \\ \hline
            First & Nein & Nein & $1\,024$ & 5,39\% \\
            First & Nein & Ja & $5\,207$ & 27,41\% \\
            First & Ja & Nein & $51$ &  0,27\% \\
            First & Ja & Ja & $1\,237$ & 6,51\% \\ \hline
            Furthest & Nein & Nein &  481 & 2,53\% \\
            Furthest & Nein & Ja & $4\,228$ & 22,25\% \\
            Furthest & Ja & Nein & 44 & 0,23\% \\
            Furthest & Ja & Ja & $1\,449$ & 7,63\% \\ \hline
            Closest & Nein & Nein & 717 & 3,77\% \\
            Closest & Nein & Ja & $3\,648$ & 19,20\% \\
            Closest & Ja & Nein & 41 & 0,22\% \\
            Closest & Ja & Ja & 873 & 4,59\% \\ \hline
        \end{tabular}
        \caption{Beste Testergebnisse nach verwendeten Algorithmen}
        \label{tab:test-results1}
    \end{table}
\end{center}
