\section{Insert-Furthest-Verfahren}
% Was ist das Insert-Furthest-Verfahren?
    % Nodes werden anhand ihrer Entfernung zur vorherigen Node in den Graphen eingefügt
    % Die am weitest entfernten werden zuerst eingefügt
% Was ist das Insert-Furthest-Verfahren
    % Zu Beginn wird wieder die Erste Node in den Pfad eingefügt, um von ihr ausgehend den restlichen Pfad zu bilden
    % anders als bei Insert First muss bzw. kann hier aber nicht auch die zweite Node eingefügt werden, da diese erst noch ermittelt werden muss
    % Spezifische Auswahl der nächsten einzufügenden Node
    % Die als nächstes eingefügt Node ist immer die, die am weitesten von der zuletzt eingefügten Node entfernt ist 
    % Allerdings muss hier vorher die Bedingung geprüft werden, ob die am weitesten Entfernte node nicht schon im Pfad ist
    % Die am weistesten Entfernte Node wird dann genau wie bei Min Dist ist den bereits existierenden Pfad an der besten verfügbaren Stelle eingefügt
    % Ermittlung der Besten Stelle, in die Node Y eingefügt werden: 
        % Es wird durch alle bereits im Graphen vorhandenen Nodes iteriert
        % Die Node der momentanen Iteration X2, ihr Vorgänger X1
        % Die Distanz berechnet sich aus der Entfernung von X1 zu Y addiert mit der Entfernung von Y zu X2
        % Es wird davon ausgegangen, dass X1 immer definiert, also teil des bereits bestehenden Graphen ist, ist 
        % X2 muss nicht zwangweise definiert sein, kann also auch null sein
        % Ist dies der fall wird für die Distanz zwischen Y und X2 0 angenommen

Aufbauend auf den Erkenntnissen des Insert-First-Verfahrens können experimentell einige Verbesserungsideen abgeleitet und ihre Auswirkungen auf das Erzeugen eines Graphen betrachtet werden.
Beim Insert-First-Verfahren wurde festgestellt, dass eine große Schwäche des Algorithmus die Reihenfolge der Betrachtung der Knoten sein kann.
Ein möglicher Ansatz, dieser in \vref{insert-first-erg} beschriebenen Schwäche entgegenzuwirken, ist die Einführung eines Kriteriums zur Betrachtung der Knoten.
Eine mögliche Umsetzung eines solchen Kriteriums ist das Insert-Furthest-Verfahren. 
Hier wird der als nächstes einzufügende Knoten ($K_i$) durch seine Distanz zum Vorgänger ($K_{i-1}$) bestimmt.

\subsection{Funktionsweise}
Ähnlich dem Insert-First-Verfahrens wird auch hier ein Graph mit einer Liste von Knoten udn einem zu Beginn leerem Pfad erzeugt.
Auch hier wird wieder der erste Knoten der Liste $K_1$ als initialer Knoten des Pfades gesetzt.
Der nächste zu betrachtende Knoten ist nun aber nicht $K_2$, sondern wird durch die Distanz zu $K_1$ bestimmt.
Ausgewählt wird der Knoten, der am weitesten von $K_1$, bzw. allgemein am weistesten von $K_{i-1}$, entfernt ist und nicht bereits Teil des Pfads ist.
Dieser Knoten wird nun auf die gleiche Weise wie die Knoten beim Insert-First-Verfahren in den Pfad des Graphen eingefügt; die Stelle mit der geringsten Distanzerhöhung für den Graphen wird gesucht und $K_i$ an dieser Stelle nach dem in \vref{code:mergeIntojava} beschriebenen Verfahren eingefügt.\\
Der Gedanke hinter der dieser Veränderung ist der Versuch Knoten mit größerer Vorraussicht als im Insert-First-Verfahren in den Pfad des Graphen einzufügen.
Dass dies zur Generierung einer besseren Route beitragen kann lässt sich am Beispiel in \vref{fig:insert-bad} erkennen.
\subsection{Ergebnis und Schwächen}