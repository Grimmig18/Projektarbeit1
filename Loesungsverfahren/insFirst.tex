\section{Insert-First-Verfahren}
% Was ist das Min Dist Verfahren?
    % Nodes werden in der Reihenfolge ihres Auftretens in den Graphen eingefügt
    % Insert-First = dadurch insert-random  
% Wie funktioniert es
    % Ein neuer Graph mit einem Pfad der länge n wird erzeugt
    % Die Nodes des Pfades des Graphen seien Y1,Y2,Y3,Y4...
    % Die erst Node des Pfads wird mit der ersten Node in der Liste der Verfügbaren Nodes befüllt (= Ausgangs-Node)
    % Die zweite Node wird ebenso aus den verfügbaren Nodes angehängt (ist nun an zweiter Stelle)
    % Nun wird durch die restlichen Verfügbaren Nodes iteriert
    % Node X sei Gegenstand des momentanen Iterationdurchlaufs 
    % Für X wird beginnend mit Y2 die Distanz zwischen Yn-1 und X + Distanz zwischen Yn und X errechnet
    % resultierend aus diesen Berechnungen wird die beste Stelle gesucht, um X in den Pfad einzufügen
% Teile des Quellcodes zeigen

\subsection{Funktionsweise}

\subsection{Ergebnisse und Schwächen}