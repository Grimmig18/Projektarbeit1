\section{Insert-Minimum-Distance-Verfahren}
    % Was ist das Min-Dist-Verfahren
        % Umgekehrtes Prinzip von Insert Furthest
        % Allerdings wird nun die nächste Node an den Graphen angehängt
    % Wie funktioniert das Min-Dist-Verfahren
        % Zuerst wird die Node in den Graphen eingefügt, welche den niedrigsten Index im Node Array hat (also am Anfang steht)
        % Dann nun wird die Node gesucht, welche am Nächsten an der ersten Node ist
        % Dazu wird einfach durch alle Nodes iteriert, die Entfernungen berechnet und dadurch die Node mit der geringsten Entfernung ermittelt
        % Diese wird dann an den Graphen angehängt, das heißt in die erste leere Stelle eingefügt
\subsection{Funktionsweise}
\subsection{Ergebnis und Schwächen}