\section{Zusammenfassung der Schwächen und Verbesserungsvorschläge}\label{sec:allBet}
% Vergleichen,
Die Ergebnisse, welche die Algorithmen erzeugen, können zum jetzigen Zeitpunkt nur begrenzt miteinander verglichen werden, da noch zu wenige Beispiele vorliegen.
Allerdings lassen sich durch die wenigen Beispiele die vorliegen bereits einige Beobachtungen treffen, anhand deren die erzeugten Graphen nachträglich noch verbessert werden können.
Betrachtet man nun diese Graphen, die durch alle drei Algorithmen erzeugt werden, so fallen bei allen schnell Schwächen auf, die ihr Ergebnis beeinträchtigen.
Anhand der vorherigen Beispiele wurde deutlich, dass die Algorithmen zwar in der Lage sind gute Ergebnisse zu erzeugen, gleichzeitig aber auch auf sich allein gestellt nicht sehr zuverlässig sind.
Dies trifft sowohl auf das Insert-First Verfahren, als auch auf die anderen vorgestellten Algorithmen zu.
\\
Bezüglich des Verhältnis' von Komplexität und Nutzen lässt sich keine genaue Aussage treffen, welcher Algorithmus effizienter arbeitet.
Alle drei Algorithmen weisen eine Komplexität von $O(n) = n^2$ und erzeugen gleichzeitig Ergebnisse, die in ihrer Qualität von optimal bis erkennbar verbesserbar reichen. 
\\
Weiterhin ist in Abbildung \vref{fig:insert-closest-BAD} ist ein Phänomen zu beobachten, welches gerade bei Anwendung der beschriebenen Algorithmen mit mehr Knoten häufig auftritt.
Das Entstehen von Überkreuzungen von Kanten zwischen zwei Knoten (siehe hierzu Abbildung \vref{app:fig:40-nodes-crossover-example}).
Gerade bei solchen Überkreuzungen besteht immer die Möglichkeit den Graph so umzulegen, dass die Gesamtdistanz sinkt.
Daher ist es sinnvoll einen Algorithmus zu entwickeln, der versucht Überkreuzungen aufzulösen.
\\
Weiterhin fallen bei manchen Graphen Konstellationen von Knoten und bestehenden Kanten auf, bei denen durch ein einfaches Umlegen der Kanten, bzw. der Knotenreihenfolge, signifikante Verringerungen in der Distanz erzielt werden können.
\\\\
% Beispielsweise erkennt man in der Veränderung von Abbildung \vref{app:subfig:40-nodes-before-after-control} zu \vref{app:subfig:40-nodes-after-after-control} ein solches Umlegen und die einhergehende Distanzverminderung.
Auch hier kann ein Algorithmus, der gezielt nach solchen Konstellationen sucht und diese aufhebt, Abhilfe schaffen.
\\\\
Nachdem diese Algorithmen im nächsten Kapitel implementiert werden folgt eine Analyse der Ergebnisse verschiedener Kombinationen der vorgestellten Algorithmen, basierend auf mehreren Testläufen mit zufällig erzeugten Knoten.
Dadurch soll es möglich sein die Algorithmen untereinander und in Verbindung miteinander zu vergleichen.
% TODO: schreiben