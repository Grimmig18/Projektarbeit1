\section{Lösungsverfahren im Vergleich}
% Vergleichen,
Betrachtet man die Graphen, die durch die drei vorgestellten Algorithmen erzeugt werden, so fallen bei allen schnell Schwächen auf, die ihr Ergebnis beeinträchtigen.
Anhand der vorherigen Beispiele wurde deutlich, dass die Algorithmen zwar in der Lage sind gute Ergebnisse zu erzeugen, gleichzeitig aber auch auf sich allein gestellt nicht sehr zuverlässig sind.
Dies trifft sowohl auf das Insert-First Verfahren, als auch auf die vorgestellten Algorithmen zu.
Weiterhin ist in Abbildung \vref{fig:insert-closest-BAD} ist ein Phänomen zu beobachten, welches gerade bei Anwendung der Algorithmen mit mehr Knoten häufig auftritt.
Das Entstehen von Überkreuzungen von Teilrouten zwischen zwei Knoten (siehe hierzu Abbildung \vref{app:fig:40-nodes-crossover-example}).
% TODO: schreiben