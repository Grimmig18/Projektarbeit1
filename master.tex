%
%   Prof. Dr. Julian Reichwald
%   auf Basis einer Vorlage von Prof. Dr. Jörg Baumgart
%   DHBW Mannheim
%
%
%	ACHTUNG: Für das Erstellen des Literaturverzeichnisses wird das modernere Paket biblatex
%			 in Kombination mit biber verwenden -- nicht mehr das ältere BibTex!
% 			 Bitte stellen Sie ggf. Ihre TeX-Umgebung
% 			 entsprechend ein (z.B. TeXStudio: Einstellungen --> Erzeugen --> Standard Bibliographieprogramm: biber)
%

\documentclass[
	12pt,
	BCOR=5mm,
	DIV=12,
	headinclude=on,
	footinclude=off,
	parskip=half,
	bibliography=totoc,
	listof=entryprefix,
	toc=listof,
	pointlessnumbers,
	plainfootsepline]{scrreprt}

%	Konfigurationsdatei einziehen
\input{config}

\begin{document}

%% BITTE GEBEN SIE HIER DEN TITEL UND DIE AUTORIN / DEN AUTOR DER ARBEIT AN!
%% DIESE INFORMATIONEN _MÜSSEN_ GESETZT SEIN, UM TITELBLATT, ABSTRACT UND
%% EIGENSTÄNDIGKEITSERKLÄRUNG AUTOMATISCH ANZUPASSEN!
\TitelDerArbeit{Analyse und Verbesserung eines Algorithmus zur heuristischen Lösung des Travelling Salesman Problems}
\AutorDerArbeit{Benno Grimm}
\Firma{SAP SE}
\Kurs{WWI18SEA}

\begin{titlepage}
\begin{minipage}{\textwidth}
		\vspace{-2cm}
		\noindent \includegraphics[scale=0.14]{img/firmenlogo.jpg} \hfill   \includegraphics{img/logo.jpg}
\end{minipage}
\vspace{1em}
\sffamily
\begin{center}
	\textsf{\large{}Duale Hochschule Baden-W\"urttemberg\\[1.5mm] Mannheim}\\[2em]
	\textsf{\textbf{\Large{}Projektarbeit I}}\\[3mm]
	\textsf{\textbf{\DerTitelDerArbeit}} \\[1.5cm]
	\textsf{\textbf{\Large{}Studiengang Wirtschaftsinformatik}\\[3mm] \textsf{Studienrichtung Software Engineering}}
	
	\vspace{3em}
	%\textsf{\Large{Sperrvermerk}}
\vfill

\begin{minipage}{\textwidth}

\begin{tabbing}
	Wissenschaftlicher Betreuer: \hspace{0.85cm}\=\kill
	Verfasser/in: \> Benno Grimm \\[1.5mm]
	Matrikelnummer: \> 5331201 \\[1.5mm]
	Firma: \> SAP SE  \\[1.5mm]
	Abteilung: \> Transportation Management \\[1.5mm]
	Kurs: \> WWI18SEA \\[1.5mm]
	Studiengangsleiter: \> Prof. Dr. Julian Reichwald  \\[1.5mm]
	Wissenschaftlicher Betreuer: \> Prof. Dr. Julian Reichwald \\
	\> julian.reichwald@dhbw-mannheim.de \\
	\> +49 (0)621 4105 - 1395 \\[1.5mm]
	Firmenbetreuer: \> Peter Wadewitz \\
	\> peter.wadewitz@sap.com \\
	\> +49 6227 7-63730 \\[1.5mm]
	Bearbeitungszeitraum: \> 13.05.2019 -- 26.08.2019
\end{tabbing}
\end{minipage}

\end{center}

\end{titlepage}

\pagenumbering{roman} % Römische Seitennummerierung
\normalfont

%--------------------------------
% Verzeichnisse - nicht benötige Verzeichnisse bitte auskommentieren / löschen.
%--------------------------------

%   Sperrvermerk
%\input{nondisclosurenotice}

%	Kurzfassung
\chapter*{Kurzfassung}
\begingroup
\begin{table}[h!]
\setlength\tabcolsep{0pt}
\begin{tabular}{p{3.7cm}p{11.7cm}}
Titel & \DerTitelDerArbeit \\
Verfasser/in: & \DerAutorDerArbeit \\
Kurs: & \DieKursbezeichnung \\
Ausbildungsstätte: & \DerNameDerFirma\\
\end{tabular}
\end{table}
\endgroup

Die Arbeit behandelt das Problem des Handlungsreisenden.
Eine Heuristik wird vorgestellt und anhand ihrer Funktionsweise und Ergebnisse analysiert.
Aufbauend auf Schwächen werden zwei Variationen dieser Heuristik entwickelt und ebenfalls analysiert.
Anschließend werden zwei Algorithmen zur nachträglichen Überarbeitung einer bereits geplanten Route vorgestellt und ebenfalls analysiert. 
Abschließend werden verschiedene Kombinationen der Algorithmen getestet.


%	Inhaltsverzeichnis
\tableofcontents

%	Abbildungsverzeichnis
\listoffigures

%	Tabellenverzeichnis
\listoftables

%	Listingsverzeichnis
 \lstlistoflistings

% 	Algorithmenverzeichnis
\listofalgorithms


% 	Abkürzungsverzeichnis (siehe Datei acronyms.tex!)
\clearpage
\chapter*{Abkürzungsverzeichnis}	
\addcontentsline{toc}{chapter}{Abkürzungsverzeichnis}


\begin{acronym}[RDBMS]
	\acro{DHBW}{Duale Hochschule Baden-Württemberg}
	\acro{RDBMS}{Relational Database Management System}
	\acro{BMBF}{Bundesministerium für Bildung und Forschung}
	\acro{TSP}{Travelling Salesman Problem}	
	\acro{LE}{Längeneinheiten}
\end{acronym}

\ohead{Acronyms} % Neue Header-Definition

%--------------------------------
% Start des Textteils der Arbeit
%--------------------------------
\clearpage
\ihead{\chaptername~\thechapter} % Neue Header-Definition (inner header)
\ohead{\headmark} % Neue Header-Definition (outer header)
\pagenumbering{arabic}  % Arabische Seitenzahlen

\chapter{Einleitung}
 \lipsum{1-3}
 \ac{DHBW}
 \ac{DHBW}
 
\chapter{Das Travelling Salesman Problem}
\section{Exakte Löungsverfahren und die Optimale Route}
\section{Heuristische Lösungsverfahren}
\chapter{L\"osungsverfahren}
\section{Insert-First-Verfahren} \label{sec:insert-first-verfahren}
% Was ist das Min Dist Verfahren?
    % Nodes werden in der Reihenfolge ihres Auftretens in den Graphen eingefügt
    % Insert-First = dadurch insert-random  
% Wie funktioniert es
    % Ein neuer Graph mit einem Pfad der länge n wird erzeugt
    % Die Nodes des Pfades des Graphen seien Y1,Y2,Y3,Y4...
    % Die erst Node des Pfads wird mit der ersten Node in der Liste der Verfügbaren Nodes befüllt (= Ausgangs-Node)
    % Die zweite Node wird ebenso aus den verfügbaren Nodes angehängt (ist nun an zweiter Stelle)
    % Nun wird durch die restlichen Verfügbaren Nodes iteriert
    % Node X sei Gegenstand des momentanen Iterationdurchlaufs 
    % Für X wird beginnend mit Y2 die Distanz zwischen Yn-1 und X + Distanz zwischen Yn und X errechnet
    % resultierend aus diesen Berechnungen wird die beste Stelle gesucht, um X in den Pfad einzufügen
% Teile des Quellcodes zeigen

Das Insert-First-Verfahren ist ein heuristischer Lösungsansatz des \ac{TSP}s, bei dem das Betrachten der Knoten zum Aufbau eines Graphen in zufälliger Reihenfolge, bzw. in der Reihenfolge ihrer Erzeugung geschieht.
Dabei wird zu einem Zeitpunkt genau ein Knoten betrachtet und an der für ihn bestmöglichen Stelle in den bereits bestehenden Graphen eingefügt.

\subsection{Funktionsweise}
Zu Beginn des Insert-First-Verfahrens wird ein neuer Graph erzeugt, welcher als Eingabewerte eine Liste mit den Knoten $K_1, K_2,  \ldots ,K_n$, hier bezeichnet als \lstinline{nodes} und der Länge $n$ erhält. 
Jeder Graph ist mit einem Pfad (oder auch \lstinline{path}) der Länge $n$ assoziiert, der die Reihenfolge der Knoten bestimmt. 
Nun wird $K_1$, der erste Knoten aus der übergebenen Liste in den Pfad des Graphs an erster Stelle eingefügt. 
Dies geschieht so oder ähnlich bei allen Verfahren, um einen statischen Ausgangspunkt zu gewährleisten und somit vergleichbare Ergebnisse zu erzielen.
Anschließend wird noch der zweite Knoten, $K_2$ angehängt.

\begin{lstlisting}[caption={Zuweisung des ersten und zweiten Knotens}]
path[0] = nodes[0];
path[1] = nodes[1];  
\end{lstlisting}

Das Vorgehen für das Einfügen der restlichen Knoten lässt sich wie folgt beschreiben: 
Sei $G$ ein Graph mit übergebener Liste von Knoten  $K_1,\ldots,K_n$ und bereits teilweise befülltem Pfad $K_1,\ldots,K_m$ mit $2 > m < n$.
Die Knoten, die noch eingefügt werden müssen, werden in der Reihenfolge ihres Auftretens in der übergebenen Liste in den Graphen eingefügt, womit der als nächstes einzufügende Knoten immer $K_{i}$ mit $i = m + 1$ ist.
\\
Um die beste Stelle zu ermitteln, in die $K_i$ eingefügt werden soll, wird für jede mögliche Stelle die Distanzerhöhung berechnet, zu der das Einfügen von $K_i$ an dieser Stelle führen würde.
Um die beste Stelle zu ermitteln, in die $K_i$ eingefügt werden soll, wird für jeden möglichen Index, also jede mögliche Stelle, die Gesamtdistanz des entstehenden Graphen berechnet. 
% Aus den so berechneten Möglichkeiten wird die mit der geringsten Distanz vermerkt und ausgewählt.
% % Um die Stelle zu ermitteln, in die $K_i$ eingefügt werden soll, müssen die Distanzen zu Vorgänger und Nachfolger berechnet und die geringste Entfernung ermittelt werden. 
% % Hierzu wird durch die Knoten beginnend mit $K_2$ bis $K_{m+1}$ iteriert. Dabei sei $K_j$ der Knoten des aktuellen Iterationdurchlaufs. 
% % Nun wird die Distanz zwischen $K_{j-1}$ und $K_i$ mit der Distanz zwischen  $K_{j}$ und $K_i$ addiert. Hierbei muss beachtet werden, dass bei der Betrachtung von $K_{j=m+1}$ keine wirkliche Distanz zu $K_i$ errechnet werden kann, da nur $m$ Knoten im Graphen sind. Stattdessen wird angenommen, dass die Distanz 0 beträgt, sodass ein Anfügen an das Ende des Graphen simuliert wird.
% \\
% In Java Quellcode bedeutet dies konkret:
% \begin{lstlisting}[caption={Ermittlung der Distanzen}, label={lst:distjava}]
% double currentDistance = 
%     ((path[j] != null) ? distances.getDistanceById(path[j], nodes[i]) : 0)
%     + distances.getDistanceById(path[j - 1], nodes[i]);

% \end{lstlisting}
Das niedrigste Ergebnis dieser Möglichkeiten wird zusammen mit dem dazugehörigen Index $j$ vermerkt. 
Nachdem die niedrigste Distanz für $K_i$ errechnet wurde kann anhand des Index' der Knoten an der bestmöglichen Stelle in den Graphen eingefügt werden. 
Einfügen bedeutet hier, dass alle Knoten, deren Index gleich oder höher $j$ ist nach hinten verschoben werden. 
Nachdem alle Knoten auf diese Weise nach hinten verschoben wurden, kann $K_i$ an der Stelle $j$ eingefügt werden, ohne, dass andere Knoten verloren gehen. 
Konkret sieht das im Java Quellcode wie folgt aus:
\begin{lstlisting}[caption={Einfügen von Knoten in einen bestehenden Graph}, label={code:mergeIntojava}]
private static Node[] mergeNodeIntoGraph(Node[] path, Node node, int index) {
    for (int i = path.length - 2; i >= index; i--) {
        path[i + 1] = path[i];
    }
    path[index] = node;
    return path;
}
\end{lstlisting}
Beispielhaft sähe das mit den vorher festgelegten Bezeichnungen wie folgt aus: 
\begin{addmargin}[1em]{2em}
\lstinline{path = } $K_1, K_2, K_4, K_3$ und $K_{i = 5}$ 
\end{addmargin}
Durch das ermitteln der Gesamtdistanzen in Abhängigkeit zu den möglichen Einfügestellen wird bekannt, dass $K_{i=5}$ mit dem Index $j=4$, also zwischen $K_4$ und $K_3$ bestmöglich eingefügt werden kann. 
Durch das Einfügen nach \vref{code:mergeIntojava} entsteht folgender Pfad:
\begin{addmargin}[1em]{2em}
\lstinline{path = } $K_1, K_2, K_4, K_5, K_3$
\end{addmargin}
für den Graphen.

\subsection{Ergebnisse und Schwächen} \label{sec:inserst-first-erg}
Bevor einige durch den Algorithmus generierte Beispiele betrachtet werden, wird hier das Szenario dieser und aller folgender Beispiele, es sei denn ist anderes angegeben, beschrieben.
Alle gezeigten Knoten befinden sich auf einem zweidimensionalem Fläche mit den Maßen zehn mal zehn \ac{LE}.
Folglich kann jedem Knoten eine X- und Y-Koordinate zwischen jeweils null und zehn zugeordnet werden. 
Dementsprechend bewegen sich auch die Gesamtdistanzen er gezeigten Graphen in dieser Größenordnung.
Weiterhin werden aus Gründen der Übersichtlichkeit für den Großteil der folgenden Beispiele nur Graphen mit fünf Knoten betrachtet.
\\\\
Bei dem Einsatz des oben beschriebenen Algorithmus kommt es zu Ergebnissen, die in ihrer Qualität nah an die optimale Lösung herankommen, teilweise aber auch weit von ihr abweichen können.  

\begin{figure}[H]
    \begin{center}
        \subfloat[$m = 2$\label{subfig:insert-first-BAD-m2}]{%
        \includegraphics[width=0.35\textwidth]{./Bilder/insertFirst/insert_first_BAD_ex_1.PNG}
        }
        \hfil
        \subfloat[$m = 3$\label{subfig:insert-first-BAD-m3}]{%
        \includegraphics[width=0.35\textwidth]{./Bilder/insertFirst/insert_first_BAD_ex_2.PNG}
        }
        \\
        \subfloat[$m = 4$\label{subfig:insert-first-BAD-m4}]{%
        \includegraphics[width=0.35\textwidth]{./Bilder/insertFirst/insert_first_BAD_ex_3.PNG}
        }
        \hfil
        \subfloat[$m = 5$\label{subfig:insert-first-BAD-m5}]{%
        \includegraphics[width=0.35\textwidth]{./Bilder/insertFirst/insert_first_BAD_ex_4.PNG}
        }
        \caption{Insert-First führt zu schlechtem Ergebnis}
        \label{fig:insert-first-bad}
    \end{center}
\end{figure}

% TODO: Alle Schritte/Bilder in den Anhang
Auf \vref{subfig:insert-first-BAD-m4} lässt sich erkennen, dass das Einfügen des vierten Knoten $K_4$ nicht optimal geschieht. 
Besser für $m = 4$ wäre hier der Pfad 
\begin{addmargin}[1em]{2em}
    \lstinline{path = } $K_1, K_3, K_2, K_4$. 
    \end{addmargin}
Dieser wird allerdings nicht durch das Insert-First-Verfahren gebildet, da dies eine Änderung des bereits erzeugten Graphen in \vref{subfig:insert-first-BAD-m3} erfordern würde. 
Dies ist jedoch nicht möglich, da $K_4$ nur zwischen bereits im Pfad des Graphen vorhandenen Knoten eingefügt werden kann, sodass der schlussendlich generierte Graph eine Gesamtlänge von 15,945 \ac{LE} hat.
Hier lässt sich auch das grundlegende Problem des Algorithmus erkennen: Das Erstellen einer Route ohne vorherige Betrachtung der Gesamtheit der Knoten. 
Einzelne Teilschritte des Graphen können gut erzeugt werden, wie beispielsweise im Schritt von \vref{subfig:insert-first-BAD-m2} zu \vref{subfig:insert-first-BAD-m3}. 
Andere hingegen, wie vorher erwähnt, nicht. 
Grund hierfür ist die alleinige Betrachtung des Knotens $K_i$. 
Spezifischer bedeutet das, dass das frühe Einfügen von Knoten in den Pfad eines Graphen später zu Komplikationen führen kann, da es objektiv besser gewesen wäre einen anderen Knoten früher einzufügen.
Am konkreten Beispiel führt die generierte Reihenfolge von 
\begin{addmargin}[1em]{2em}
    \lstinline{path = } $K_1, K_2, K_3$. 
    \end{addmargin}
in \vref{subfig:insert-first-BAD-m3} dazu, dass $K_4$ nur unter einen vergleichsweise großen Gesamtdistanzzuwachs in den Graphen eingefügt werden kann. 
\\\\
Konträr zu diesem schlechten Beispiel ist der Insert-First-Algorithmus auch in der Lage gute bis optimale Ergebnisse zu generieren. 

\begin{figure}[H]
    \begin{center}
        \subfloat[$m = 2$\label{subfig:insert-first-GOOD-m2}]{%
        \includegraphics[width=0.35\textwidth]{./Bilder/insertFirst/insert_first_ex_1.PNG}
        }
        \hfil
        \subfloat[$m = 3$\label{subfig:insert-first-GOOD-m3}]{%
        \includegraphics[width=0.35\textwidth]{./Bilder/insertFirst/insert_first_ex_2.PNG}
        }\\
        \subfloat[$m = 4$\label{subfig:insert-first-GOOD-m4}]{%
        \includegraphics[width=0.35\textwidth]{./Bilder/insertFirst/insert_first_ex_3.PNG}
        }
        \hfil
        \subfloat[$m = 5$\label{subfig:insert-first-GOOD-m5}]{%
        \includegraphics[width=0.35\textwidth]{./Bilder/insertFirst/insert_first_ex_4.PNG}
        }
        \caption{Insert-First führt zu guten Ergebnis}
        \label{fig:insert-first-good}
    \end{center}
\end{figure}
% TODO: Alle Schritte/Bilder in den Anhang

Am Beispiel in \vref{fig:insert-first-good} lässt sich erkennen, wie der Insert-First-Algorithmus einen optimalen Pfad mit den gegebenen Knoten generiert. 
Gerade im Schritt von \vref{subfig:insert-first-GOOD-m4} zu \vref{subfig:insert-first-GOOD-m5} ist ein funktionierendes und korrektes Einfügen des Knotens in den Graphen zu sehen, bei dem der Anstieg der Gesamtdistanz der Route sehr gering gehalten wird. 
Hier wird der aktuelle Knoten mit geringem Zuwachs der schlussendlichen Gesamtdistanz in den Graphen eingefügt, sodass die Gesamtdistanz zum Ende bei 16,691 \ac{LE} liegt.
Die scheint zwar höher als das vorherige schlechte beispiel, das liegt aber an den Positionen der einzelnen Knoten.
\\\\
Für die Bewertung des Algorithmus müssen also beide Seiten betrachtet werden. 
Zwar ist Insert-First in der Lage eine gute oder auch optimale Route zu erstellen, allerdings beeinflusst die Reihenfolge der Betrachtung der Knoten stark die Qualität des Endergebnisses.
\section{Insert-Furthest-Verfahren} \label{sec:insert-furthest-verfahren}
% Was ist das Insert-Furthest-Verfahren?
    % Nodes werden anhand ihrer Entfernung zur vorherigen Node in den Graphen eingefügt
    % Die am weitest entfernten werden zuerst eingefügt
% Was ist das Insert-Furthest-Verfahren
    % Zu Beginn wird wieder die Erste Node in den Pfad eingefügt, um von ihr ausgehend den restlichen Pfad zu bilden
    % anders als bei Insert First muss bzw. kann hier aber nicht auch die zweite Node eingefügt werden, da diese erst noch ermittelt werden muss
    % Spezifische Auswahl der nächsten einzufügenden Node
    % Die als nächstes eingefügt Node ist immer die, die am weitesten von der zuletzt eingefügten Node entfernt ist 
    % Allerdings muss hier vorher die Bedingung geprüft werden, ob die am weitesten Entfernte node nicht schon im Pfad ist
    % Die am weistesten Entfernte Node wird dann genau wie bei Min Dist ist den bereits existierenden Pfad an der besten verfügbaren Stelle eingefügt
    % Ermittlung der Besten Stelle, in die Node Y eingefügt werden: 
        % Es wird durch alle bereits im Graphen vorhandenen Nodes iteriert
        % Die Node der momentanen Iteration X2, ihr Vorgänger X1
        % Die Distanz berechnet sich aus der Entfernung von X1 zu Y addiert mit der Entfernung von Y zu X2
        % Es wird davon ausgegangen, dass X1 immer definiert, also teil des bereits bestehenden Graphen ist, ist 
        % X2 muss nicht zwangweise definiert sein, kann also auch null sein
        % Ist dies der fall wird für die Distanz zwischen Y und X2 0 angenommen
\subsection{Funktionsweise} \label{sec:insert-furthest-funkt}
Aufbauend auf den Erkenntnissen des Insert-First-Verfahrens können experimentell einige Verbesserungsideen abgeleitet und ihre Auswirkungen auf die Erzeugung eines Pfads betrachtet werden.
Bei Betrachtung der Ergebnisse des Insert-First-Algorithmus wurde festgestellt, dass eine große Schwäche des Algorithmus die Reihenfolge der Betrachtung der Knoten sein kann.
Ein möglicher Ansatz, dieser in \vref{sec:inserst-first-erg} beschriebenen Schwäche entgegenzuwirken, ist die Einführung eines Kriteriums für eben diese Reihenfolge der Betrachtung der Knoten.
Eine mögliche Umsetzung eines solchen Kriteriums ist das Insert-Furthest-Verfahren. 
Hier wird der als nächstes einzufügende Knoten ($k_i$) durch seine Distanz zum Vorgänger ($k_{i-1}$) bestimmt.


% \subsection{Funktionsweise}

Ähnlich dem Insert-First-Verfahren wird auch hier ein Graph mit einer Liste von Knoten und einem zu Beginn leerem Pfad erzeugt.
Auch hier wird wieder der erste Knoten der Liste $k_1$ als initialer Knoten $p_1$ des Pfades $P$ gesetzt.
Der nächste zu betrachtende Knoten ist nun aber nicht $k_2$, sondern wird durch die Distanz zu $k_1$ bestimmt.
Ausgewählt wird der Knoten, der am weitesten von $k_1$, bzw. allgemein am weistesten von $k_{i-1}$, entfernt ist und nicht bereits Teil des Pfads ist.
Dieser Knoten wird nun auf die gleiche Weise wie die Knoten beim Insert-First-Verfahren in den Pfad des Graphen eingefügt; die Stelle mit der geringsten Distanzerhöhung für den Graphen wird ermittelt und $k_i$ an dieser Stelle nach dem im Algorithmus \vref{alg:merge-node-into-path} beschriebenen Verfahren eingefügt.\\
Der Gedanke hinter der dieser Veränderung ist der Versuch Knoten mit größerer Vorraussicht als im Insert-First-Verfahren in den Pfad einzufügen.
Ziel ist es mit den ersten paar Knoten einen Pfad zu generieren, der einen großen Teil der Fläche überspannt, auf der sich Knoten befinden.
Das kann insofern zu einem besserem Ergebnis führen, dass die ersten Knoten zwar unter einer -- relativ zur schlussendlichen Gesamtlänge des Graphen -- hohen Distanzerhöhung eingefügt werden, die nachfolgenden Knoten aber durch geringe Umwege des bestehenden Pfads in den Graphen eingebunden werden können.
Auf diese Weise sollen Komplikationen beim Einfügen der letzten Knoten verhindert werden und so suboptimale Graphen wie in \vref{fig:insert-first-bad} umgangen werden. Algorithmus \vref{alg:insert-furthest} zeigt eine beispielhafte Umsetzung des Algorithmus in Pseudocode.

\begin{algorithm}[H]
    \caption{Insert-Furthest-Algorithmus}
    \label{alg:insert-furthest}
    \begin{algorithmic}[1]
        \Require Graph $G$, Pfad $P$ 
        \Require $G=(K,E),K= k_1,k_2,\ldots,k_n$, $n > 2$ 
        % \Require $P=p_1,\cdots,p_m$, $\forall p = k_G$
        \State $p_1 \gets k_1$
        \Comment Setzen des ersten Knoten
        % Find furthest
        \State $i \gets -1$
        \State $d_i \gets -1$

        \For{$a \gets 1$, $a \leq n$, $a \gets a + 1$}
            % Main Loop
            % find Furthest from last
            \State $j_F \gets -1$
            \State $d_F \gets -1$

            \For{$b \gets 2$, $b \leq n$, $b \gets b + 1$}
                \Comment Finde $k_b$ mit der höchsten Distanz zu $p_m$
                \State $d_C \gets \omega$($k_b$, $p_m$)
                \Comment Distanz zwischen $k_b$ und letztem Knoten $p_m$
                \If{$k_b \not \in P$ \textbf{and} ($j_F = -1$ \textbf(or) $d_C > d_F$)}
                    \State $d_F \gets d_C$
                    \State $j_F \gets b$
                \EndIf
            \EndFor

            \State $i_S \gets -1$
            \State $d_S \gets -1$

            \For{$b \gets 2$, $b < m$, $b \gets b + 1$}
                \State $d_C \gets$ \textsc{mergeAt}($P$, $b$, $k_{j_F}$) \textsc{distance}
                \Comment Gesamtlänge des entstehenden Pfads
                \If{$i_S = -1$ \textbf{or} $d_C < d_S$}
                    \State $i_S \gets b$
                    \State $d_S \gets d_C$
                \EndIf
            \EndFor
            \State $P \gets$ \textsc{mergeAt}($P$, $k_{j_F}$, $i_S$)
            \Comment Siehe Alg. \vref{alg:merge-node-into-path}
        \EndFor \\
        \Return new Graph($P$)
    \end{algorithmic}
\end{algorithm}

\subsection{Zeitkomplexität}
Der Insert-Furthest-Algorithmus fügt verglichen mit dem Insert-First-Algorithmus ein Auswahlkriterium hinzu.
Dieses Kriterium drückt sich im Pseudocode im Algorithmus \vref{alg:insert-furthest} durch eine zusätzliche Schleife in Zeile vier aus.
Um nun die Zeitkomplexität zu ermitteln, reicht es die Laufzeit dieser Schleife, $n^2$, auf die in \vref{sec:time-comp-first} berechnete zu addieren, wodurch sich 
$$f(n) = \frac{n^2+n}{2}+n^2 -1$$
und dadurch auch hier eine Komplexität von
$$f(n) = O(n^2)$$
ergibt.
Damit skaliert dieser Algorithmus bei sich verändernder Eingabe in etwa genauso wie Insert-First.

\subsection{Ergebnis und Schwächen}
Testet man des Insert-Furthest-Verfahren anhand der Knoten des Beispiels \vref{fig:insert-first-bad} wird sichtbar, dass der Algorithmus tatsächlich in der Lage ist einen besseren Pfad zu generieren als das Insert-First-Verfahren.

\begin{figure}[H]
    \begin{center}
        % \subfloat[$m = 2$\label{subfig:insert-first-BAD-m2}]{%
        % \includegraphics[width=0.35\textwidth]{./Bilder/insertFurthest/insert_furthest_ex_1.PNG}
        % }
        % \hfil
        \subfloat[$m = 2$\label{subfig:insert-furthest-GOOD-m2}]{%
        \includegraphics[width=0.35\textwidth]{./Bilder/insertFurthest/insert_furthest_ex_2.PNG}
        }
        % \\
        % \subfloat[$m = 4$\label{subfig:insert-first-BAD-m4}]{%
        % \includegraphics[width=0.35\textwidth]{./Bilder/insertFurthest/insert_furthest_ex_3.PNG}
        % }
        \hfil
        \subfloat[$m = 5$\label{subfig:insert-furthest-GOOD-m5}]{%
        \includegraphics[width=0.35\textwidth]{./Bilder/insertFurthest/insert_furthest_ex_5.PNG}
        }
        \caption{Insert-First führt zu einem gutem Ergebnis}
        \label{fig:insert-furthest-good}
    \end{center}
\end{figure}
Der vollständige Generierungsvorgang befindet sich im Anhang (Abbildung A.3 im Anhang). \\

% TODO: Alle Schritte/Bilder in den Anhang
In \vref{subfig:insert-furthest-GOOD-m2} ist zu erkennen, dass, anstatt wie in \vref{subfig:insert-first-BAD-m2} $k_2$, $k_4$ als erster Knoten eingefügt wird.
Dieses Verhalten ist nach der in \vref{sec:insert-furthest-funkt} definierten Funktionsweise zu erwarten, da $k_4$ der Knoten mit der größten Distanz zu $k_1$ ist und daher als erstes in den Pfad des Graphen eingefügt wird.
Nach der Ausführung aller Schritte des Algorithmus wird der in \vref{subfig:insert-furthest-GOOD-m5} zu sehenden Graphen generiert.
Dieser hat eine Gesamtdistanz von 13,327 \ac{LE} und ist somit im Vergleich mit dem in \vref{fig:insert-first-bad} durch das Insert-First-Verfahren erzeugten Graph 2,618 \ac{LE} oder 16,41\% kürzer.
\\\\
Ebenso wie das Insert-First-Verfahren kann das Insert-Furthest-Verfahren auch Graphen generieren die in ihrer Gesamtdistanz vom Optimum abweichen. Am folgenden Beispiel wird deutlich, dass auch dieser Algorithmus von ähnlichen Schwächen betroffen ist.
\begin{figure}[H]
    \begin{center}
        % \subfloat[$m = 2$\label{subfig:insert-first-BAD-m2}]{%
        % \includegraphics[width=0.35\textwidth]{./Bilder/insertFurthest/insert_furthest_ex_1.PNG}
        % }
        % \hfil
        \subfloat[$m = 4$\label{subfig:insert-furthest-BAD-m2}]{%
        \includegraphics[width=0.35\textwidth]{./Bilder/insertFurthest/insert_furthest_ex_BAD_4.PNG}
        }
        % \\
        % \subfloat[$m = 4$\label{subfig:insert-first-BAD-m4}]{%
        % \includegraphics[width=0.35\textwidth]{./Bilder/insertFurthest/insert_furthest_ex_3.PNG}
        % }
        \hfil
        \subfloat[$m = 5$\label{subfig:insert-furthest-BAD-m5}]{%
        \includegraphics[width=0.35\textwidth]{./Bilder/insertFurthest/insert_furthest_ex_BAD_5.PNG}
        }
        \caption{Insert-First führt zu einem schlechtem Ergebnis}
        \label{fig:insert-furthest-bad}
    \end{center}
\end{figure}
% TODO: Bilder in den Anhang
Der vollständige Generierungsvorgang befindet sich im Anhang (Abbildung A.4 im Anhang). \\
\\\\
Auch hier lässt sich das Problem der Reihenfolge beobachten, welches in \vref{sec:inserst-first-erg} beschrieben wurde.
Durch das Einfügen des letzten Knotens $k_4$ in den Pfad entsteht ein suboptimaler Graph.
Werden die Distanzen zwischen den Knoten $k_4$ und $k_5$ (6,129 \ac{LE}) mit den zwischen $k_5$ und $k_2$ (5,019 \ac{LE}) verglichen, wird schnell ersichtlich, dass eine Verminderung der Distanz durch das Umlegen der Knoten erreicht werden kann.
Verursacht wird diese Abweichung vom Optimum dadurch, dass $k_4$ als letztes in den Graphen eingefügt wird, da der Knoten sich, relativ zu den restlichen Knoten, in der Mitte der Fläche befindet und somit aufgrund seiner geringeren Entfernung zu anderen Knoten vom Algorithmus als letztes betrachtet wird.
Die optimale Route
% \begin{addmargin}[1em]{2em}
$$P = k_1, k_3, k_4, k_2, k_5$$ 
    % \end{addmargin}
 würde es jedoch erfordern, dass $k_4$ früher betrachtet und in den Pfad eingefügt wird.
Der durch den Algorithmus erzeugte Graph hat eine Gesamtlänge von 21,138 \ac{LE}, während durch das Umlegen zum optimalen Graph eine Länge von 20,028 \ac{LE}, also Reduktion der Distanz um 1,11 \ac{LE} oder 5,251\% erreicht werden kann.
\\\\
Das Insert-Furthest-Verfahren verhält sich in einigen Fällen, wie in \vref{fig:insert-furthest-good} gezeigt, besser als das Insert-First-Verfahren, weist aber immer noch eindeutige Schwächen, gerade im Bezug auf die Reihenfolge der Betrachtung der Knoten, auf.
Das Beispiel in \vref{fig:insert-furthest-bad} zeigt deutlich, wie auch hier die Positionierung der Knoten einen negativen Einfluss auf das Endergebnis hat.
\section{Insert-Closest-Verfahren}
    % Was ist das Min-Dist-Verfahren
        % Umgekehrtes Prinzip von Insert Furthest
        % Allerdings wird nun die nächste Node an den Graphen angehängt
    % Wie funktioniert das Min-Dist-Verfahren
        % Zuerst wird die Node in den Graphen eingefügt, welche den niedrigsten Index im Node Array hat (also am Anfang steht)
        % Dann nun wird die Node gesucht, welche am Nächsten an der ersten Node ist
        % Dazu wird einfach durch alle Nodes iteriert, die Entfernungen berechnet und dadurch die Node mit der geringsten Entfernung ermittelt
        % Diese wird dann an den Graphen angehängt, das heißt in die erste leere Stelle eingefügt
Aufbauend auf den durch das Insert-First- und Insert-Furthest-Verfahren gewonnen Erkenntnissen ist es mögliche weitere Variationen der Heuristik zu entwickeln und deren Ergebnisse zu betrachten.
Da sowohl in \vref{sec:insert-first-verfahren} als auch in \vref{sec:insert-furthest-verfahren} festgestellt wurde, dass ein Grund für suboptimal erstellte Routen die Reihenfolge der Betrachtung der Knoten ist, wird für das Insert-Closest-Verfahren eine weitere anderes Kriterium für eben diese Reihenfolge festgelegt.
Wie der Name des Verfahrens schon suggeriert, geschieht hier die Auswahl der Knoten wieder nach ihrer Distanz.

\subsection{Funktionsweise}
Das Insert-Closest-Verfahren bezeichnet im Grundprinzip die Umkehrung des Insert-Furthest-Prinzips. 
Anstatt des am weitesten entfernten Knotens wird hier der dem aktuellen Knoten nächste betrachtet.
\\
Zu Beginn beschreibt sich der Algorithmus identisch zum Insert-Furthest-Verfahren.
Auch hier wird ein neuer Graph mit einem leerem Pfad \lstinline{path} und einer Liste von Knoten $K_1, \dots ,K_n$ der Länge $n$ erzeugt.
Auch hier wird der erste Knoten $K_1$ als erster Knoten des Pfades festgelegt.
Der als nächstes einzufügende Knoten wird wie beim Insert-Furthest-Verfahren durch seine Distanz zum vorherigen bestimmt.
Für das Insert-Closest-Verfahren wird der dem Vorherigen Knoten nächste Knoten, unter der Bedingung, dass dieser nicht bereits Teil des Graph ist, in den Pfad eingefügt.
Um die beste Stelle zum Einfügen des Knotens zu ermitteln wird das gleiche Verfahren wie bei den beiden vorherigen Algorithmen angewandt; die Stelle, die den geringsten Anstieg für die Gesamtdistanz des Pfads wird ausgewählt.
Auch das Vorgehen beim Einfügen orientiert sich hier an den vorherigen Algorithmen.
Der aktuelle Knoten $K_i$ wird nach dem in \vref{code:mergeIntojava} dargestelltem Prinzip an dem vorher festgelegten Index $j$ in den Graph eingefügt.

\subsection{Ergebnis und Schwächen}
Auch die Ergebnisse dieses Algorithmus gestalten sich sehr divers.
Wie bei den beiden anderen Verfahren entstehen hier Graphen, die nahe an die Optimale Route heranreichen oder ihr teilweise auch entsprechen, aber auch solche, die weit von der optimalen Lösung abweichen.
Übergibt man dem Algorithmus beispielsweise die gleichen Knoten wie in \vref{fig:insert-first-bad}, einem Szenario, bei dem der Insert-First Algorithmus einen suboptimalen Graph generiert, kommt der Insert-Closest Algorithmus, so wie der Insert-Furthest Algorithmus auch, auf die optimale Route.\\
So wie die anderen Algorithmen stößt auch der Insert-Closest Algorithmus bei bestimmten Konstellationen von Knoten an seine Grenzen.

\begin{figure}
    \begin{center}
        \subfloat[$m=4$ \label{subfig:insert-closest-BAD-4}]{
            \includegraphics[width=0.35\textwidth]{Bilder/insertClosest/insert_closest_ex_BAD_4.PNG}
        }
        \hfil
        \subfloat[$m=5$ \label{subfig:insert-closest-BAD-5}]{
            \includegraphics[width=0.35\textwidth]{Bilder/insertClosest/insert_closest_ex_BAD_5.PNG}
        }
        \caption{Der Insert-Closest Algorithmus kommt zu einem schlechten Ergebnis}
        \label{fig:insert-closest-BAD}
    \end{center}
\end{figure}

Wie in \vref{fig:insert-closest-BAD} zu sehen ist, plant der Algorithmus einen Pfad, der deutlich erkennbar nicht optimal ist.
Die einzelnen Schritte, die zur Generierung des Graphen führen finden sich im Anhang in der Abbildung \vref{app:fig:insert-closest-BAD-complete}.
Der in \vref{subfig:insert-closest-BAD-5} gezeigt Pfad hat eine Gesamtdistanz von 15,336 \ac{LE}.
Durch das Ändern der Reihenfolge der Knoten zu
\begin{addmargin}[1em]{2em}
    \lstinline{path = } $K_1, K_4, K_5, K_2, K_3$ 
\end{addmargin}
könnte eine Verringerung der Distanz im Vergleich zum vorherigen Resultat um 3,208 \ac{LE}, bzw. 20,918\% erreicht werden.
Auch hier wird der suboptimal geplante Pfad durch die Reihenfolge der Betrachtung der Knoten verursacht.
Im konkreten Beispiel werden die ersten vier Knoten $K_1, K_2, K_4$ und $K_5$ zu einem für sich optimalen Pfad zusammengefügt.
$K_3$ wird aufgrund seiner hohen Distanz zu den übrigen Knoten als letztes in den Pfad eingefügt.
Das hat hier zur Folge, dass $K_3$, als letzter Knoten des Pfads, nach dem von ihm am weitesten entfernten Knoten eingefügt wird, was zu einem hohen Zuwachs der Gesamtdistanz führt.
\section{Zusammenfassung der Schwächen und Verbesserungsvorschläge}\label{sec:allBet}
% Vergleichen,
Die Ergebnisse, welche die Algorithmen erzeugen, können zum jetzigen Zeitpunkt nur begrenzt miteinander verglichen werden, da noch zu wenige Beispiele vorliegen.
Allerdings lassen sich durch die wenigen Beispiele die vorliegen bereits einige Beobachtungen treffen, anhand deren die erzeugten Graphen nachträglich noch verbessert werden können.
Betrachtet man nun diese Graphen, die durch alle drei Algorithmen erzeugt werden, so fallen bei allen schnell Schwächen auf, die ihr Ergebnis beeinträchtigen.
Anhand der vorherigen Beispiele wurde deutlich, dass die Algorithmen zwar in der Lage sind gute Ergebnisse zu erzeugen, gleichzeitig aber auch auf sich allein gestellt nicht sehr zuverlässig sind.
Dies trifft sowohl auf das Insert-First Verfahren, als auch auf die anderen vorgestellten Algorithmen zu.
\\
Bezüglich des Verhältnis' von Komplexität und Nutzen lässt sich keine genaue Aussage treffen, welcher Algorithmus effizienter arbeitet.
Alle drei Algorithmen weisen eine Komplexität von $O(n^2)$ auf und erzeugen Ergebnisse, die in ihrer Qualität von optimal bis erkennbar verbesserbar reichen. 
\\
Weiterhin ist in Abbildung \vref{fig:insert-closest-BAD} ein Phänomen zu beobachten, welches gerade bei Anwendung der beschriebenen Algorithmen mit mehr Knoten häufig auftritt.
Das Entstehen von Überkreuzungen von Kanten zwischen zwei Knoten (siehe hierzu Abbildung \vref{app:fig:40-nodes-crossover-example}).
Gerade bei solchen Überkreuzungen besteht immer die Möglichkeit den Graph so umzulegen, dass die Gesamtdistanz sinkt.
Daher ist es sinnvoll einen Algorithmus zu entwickeln, der versucht Überkreuzungen aufzulösen.
\\
Weiterhin fallen bei manchen Graphen Konstellationen von Knoten und bestehenden Kanten auf, bei denen durch ein einfaches Umlegen der Kanten, bzw. der Knotenreihenfolge, signifikante Verringerungen in der Distanz erzielt werden können (siehe hierzu Abbildung \vref{app:subfig:40-nodes-before-after-control}).
\\\\
% Beispielsweise erkennt man in der Veränderung von Abbildung \vref{app:subfig:40-nodes-before-after-control} zu \vref{app:subfig:40-nodes-after-after-control} ein solches Umlegen und die einhergehende Distanzverminderung.
Auch hier kann ein Algorithmus, der gezielt nach solchen Konstellationen sucht und diese aufhebt, Abhilfe schaffen.
\\\\
Nachdem diese Algorithmen im nächsten Kapitel erarbeitet werden folgt eine Analyse der Ergebnisse verschiedener Kombinationen der vorgestellten Algorithmen, basierend auf mehreren Testläufen mit zufällig erzeugten Knoten.
Dadurch soll es möglich sein die Algorithmen untereinander und in Verbindung miteinander zu vergleichen.
\chapter{Überarbeitung eines bestehenden Pfads}

\section{Entfernen von Überschneidungen}
\subsection{Beschreibung des Problems}
Wie in Abschnitt \vref{sec:allBet} angedeutet kommt es kommt es bei generierten Graphen zu Überkreuzungen von Teilrouten zwischen jeweils zwei Knoten.
Betrachtet man solche Überkreuzungen im Detail fallen einige Gemeinsamkeiten zwischen ihnen auf.
So ist es beispielsweise immer möglich eine Überkreuzung durch das Verändern der Reihenfolge der Knoten im Pfad aufzulösen und so eine Verringerung in der Gesamtdistanz zu erreichen.

\begin{figure}[h]
    \begin{center}
        \subfloat[Graph mit einer Überkreuzung\label{subfig:graph-with-crossover-and-rect}]{
            \includegraphics[width=0.35\textwidth]{Bilder/crossover/7_nodes_with_crossover.PNG}
        }
        \hfil
        \subfloat[Aufgelöste Überkreuzung\label{subfig:graph-without-crossover}]{
            \includegraphics[width=0.35\textwidth]{Bilder/crossover/7_nodes_without_crossover.PNG}
        }
        \caption{Graph mit und ohne Überkreuzung (Das rote Rechteck in Abbildung a) dient späteren Illustrationszwecken)}\label{fig:graph-with-and-without-crossover}
    \end{center}
\end{figure}

Anhand dieses Beispiels wird nun das Entstehen, Erkennen und Auflösen von Überkreuzungen erläutert.

% Allgemein
Eine Überkreuzung repräsentiert das Auftreten eines Schnittpunkts von zwei Kanten eines Graphen in einem für den Graphen relevanten Bereich.
Ein Schnittpunkt von zwei Kanten bedeutet hier, dass keine der vier Knoten gleich sein dürfen.
Besteht eine Überkreuzung also aus den Knotenpaaren $A$ und $B$ mit $A=K_{A_1},K_{A_2}$ und $B=K_{B_1},K_{B_2}$, dann muss gelten $K_{A_1} \neq K_{A_2} \neq K_{B_1} \neq K_{B_2}$.
Ist diese Bedingung nicht erfüllt, kann es keine Überkreuzung geben.
Sind nur drei der vier benötigten Knoten einzigartig kann es nicht zu einer Überkreuzung kommen, da dies einen zusammenhängenden Streckenabschnitt der Form $K_1, K_2, K_3$ darstellen würde.

\subsection{Erkennnen von Überkreuzungen}
Um Überkreuzungen erkennen zu können ist die Bedingung \enquote{in einem für den Graphen relevanten Bereich} wichtig.
Betrachtet man zwei zufällig ausgewählte Kanten als unendliche Linien, stellt man sie also in der zweidimensionalen Ebene als lineare Funktion, mit einer Steigung und einem Schnittpunkt mit der Ordinate, dar, dann schneiden sich alle diese Funktionen an irgendeinem Punkt, es sei denn sie sind parallel zueinander.
Um zu überprüfen, ob eine Überkreuzung im für die Generierung eines Graphen relevanten Bereich ist, wird zuerst der Schnittpunkt der beiden Kanten berechnet.
Dazu wird aus zwei Knoten einer Kante eine lineare Funktion der Form $f(x) =mx+n$ simuliert, wobei $m=\frac{\Delta y}{\Delta x}$ und $n=y - mx$.
Sind die Knoten der beiden Kanten nun $A_1,A_2$ und $B_1,B_2$, dann ergibt sich für die Berechnung der Schnittstelle:
\begin{equation}
    \label{eq:calculation-xs}
    % x_S = (y_{B_1} - (x_{B_1}\cdot \frac{(y_{A_2} - y_{A_1})}{(x_{A_2} - xy_{A_1})}))
    x_S = \frac{(y_{B_1} - \frac{y_{B_2} - y_{B_1}}{x_{B_2} - x_{B_1}}\cdot x_{B_1}) - (y_{A_1} - \frac{y_{A_2} - y_{A_1}}{x_{A_2} - x_{A_1}}\cdot x_{A_1})}{(\frac{y_{A_2} - y_{A_1}}{x_{A_2} - x_{A_1}}) - (\frac{y_{B_2} - y_{B_1}}{x_{B_2} - x_{B_1}})} 
\end{equation}
\begin{equation}
    \label{eq:calculation-ys}
    y_S = \frac{y_{A_2} - y_{A_1}}{x_{A_2} - x_{A_1}}\cdot x_S + (y_{A_1} - \frac{y_{A_2} - y_{A_1}}{x_{A_2} - x_{A_1}}\cdot x_{A_1})
\end{equation}
Mit $x_S$ und $y_S$ lässt sich der Punkt $S(x_S|y_S)$ konstruieren.
\\\\
Mit Hilfe des Punkts $S$ gilt es nun zu überprüfen, ob sich dieser im relevanten Bereich befindet.
Um dies zu bestimmen wird um die Knoten beider Kanten jeweils ein Recht simuliert, wie es beispielhaft in Abbildung \vref{subfig:graph-with-crossover-and-rect} eingezeichnet ist.
Eine Überkreuzung ist genau dann für den Algorithmus relevant, wenn sie in den Rechtecken beider Kanten liegt.
Um dies zu überprüfen wird folgender Algorithmus angewandt:
\begin{algorithm}[H]
    \caption{Erkennen von Überkreuzungen}
    \label{alg:check-point-in-rect}
    \begin{algorithmic}[1]
        \Require Knoten $A_1,A_2$, Punkt $P$
        \State score $\gets 0$
        \If{$x_{A_1} > x_{A_2}$}
            \If{$x_P > x_{A_2}$ \textbf{and} $x_P < x_{A_1}$}
                \State score $\gets$ score $+ 1$
            \EndIf
        \Else
            \If{$x_P < x_{A_2}$ \textbf{and} $x_P > x_{A_1}$}
                \State score $\gets$ score $+ 1$
            \EndIf
        \EndIf

        \If{$y_{A_1} > y_{A_2}$}
            \If{$y_P < y_{A_1}$ \textbf{and} $y_P > y_{A_2}$}
                \State score $\gets$ score $+ 1$
            \EndIf
        \Else
            \If{$y_P > y_{A_1}$ \textbf{and} $y_P < y_{A_2}$}
                \State score $\gets$ score $+ 1$
            \EndIf
        \EndIf\\
        \Return (score $== 2$)
        \Comment score $==2$ gibt wahr zurück und signalisiert, dass $P$ im Rechteck von $A_1$ und $A_2$ ist
    \end{algorithmic}
\end{algorithm}
An dieser Stelle sei angemerkt, dass das Problem der Überkreuzungserkennung auch mit Hilfe von Vektorenskalierung lösbar ist.
Dieses Verfahren wird jedoch in dieser Arbeit nicht diskutiert.


\subsection{Auflösen von Überkreuzungen}

Um einen Algorithmus zur Auflösung von Überkreuzungen entwickeln zu können, ist es wichtig die Knoten um eine Überkreuzung herum vor und nach deren Auflösung zu betrachten.
Dazu kann als Beispiel wieder Abbildung \vref{fig:graph-with-and-without-crossover} dienen.
Reihenfolge der Knoten in Abbildung \vref{subfig:graph-with-crossover-and-rect} ist $$P_{alt} = K_1, K_2, K_3, K_4, K_5, K_6, K_7$$
Die Reihenfolge der Knoten nach dem Auflösen in Abbildung \vref{subfig:graph-without-crossover} ist $$P_{neu} = K_1, K_6, K_5, K_4, K_3, K_2, K_7$$
In diesem Beispiel seien die betroffenen Kanten $A$ und $B$ mit den Knoten $A_1 = K_1$, $A_2 = K_2$ und $B_1 = K_6$, $B_2 = K_7$.
Die hier interessanten Knoten sind $K_2$ bis $K_6$, da sich deren Reihenfolge umkehrt.
Daraus kann gefolgert werden, dass zum Auflösen einer Überkreuzung das Umkehren der betroffenen Knoten in der Mitte reicht.
Diese betroffenen Knoten bestimmen sich durch die Kanten $A$ und $B$ -- der erste umzukehrende Knoten ist immer $A_2$, während der letzte $B_1$ ist.
Dies ist allerdings nur unter der Bedingung wahr, dass $A$ im Graph vor $B$ ist.
% Ist dies nicht der Fall müssen $A$ und $B$ getauscht werden, sodass 
Ist dies nicht der Fall kehren sich die Rollen der Knoten um und $B_1$ ist der erste, während $A_2$ der letzte umzukehrende Knoten ist.
Ein Algorithmus, der auf einem Pfad mit Überkreuzung und bekannten $A$ und $B$ eben dieses Tauschen ausführt, findet sich im Anhang unter Algorithmus \vref{alg:swap-nodes-inbetween}.
\\\\
% Ein vollständiger Algorithmus, der die aufgeführten Methodiken anwendet um Überkreuzungen zu erkennen und aufzulösen könnte beispielhaft
Ein vollständiger Algorithmus, der die aufgeführten Methodiken anwenden soll, um Überkreuzungen zu erkennen und aufzulösen, müssen also alle Kanten gegeneinander geprüft werden.
Eine simple Umsetzung davon, die alle beschriebenen Methodiken mit einschließt kann sein
\begin{algorithm}
    \caption{Erkennen und Auflösen von Überkreuzungen auf einem Pfad}
    \label{alg:handle-crossover}
    \begin{algorithmic}[1]
        \Require Pfad $P$
        \Require $P=p_1,p_2\cdots,p_n$, $n \geq 4$, $\forall p \in G$
        
        \For{$a \gets 2$, $a \leq n$, $a \gets a + 1$}
            \For{$b \gets b$, $b \leq n$, $b \gets b + 1$}
                \If{\textbf{not} ($p_a \neq p_{b} \textrm{\textbf{ and }} p_a \neq p_{b-1} \textrm{\textbf{ and }} p_{a-1} \neq p_b \textrm{\textbf{ and }} p_{a-1} \neq p_{b-1}$}
                    \State \textsc{continue}
                \EndIf
                \State $x_S \gets $ nach \vref{eq:calculation-xs}
                \Comment $p_a$ und $p_{a-1}$ entsprechen $A_1$ und $A_2$ 
                \State $y_S \gets $ nach \vref{eq:calculation-ys}
                \Comment $p_b$ und $p_{b-1}$ entsprechen $B_1$ und $B_2$ 
                \If{\textsc{check}($p_a,p_{a-1},S(x_S|y_S)$) \textbf{and} \textsc{check}($p_b,p_{b-1},S(x_S|y_S)$)} 
                \Comment \textsc{check} repräsentiert dabei \ac{Alg.} \vref{alg:check-point-in-rect}
                    \State $P \gets $ \textsc{resolve}($P$, $p_a$, $p_{b-1}$)
                    \Comment Auflösen der Überkreuzung mit \ac{Alg.} \vref{alg:swap-nodes-inbetween}
                \EndIf
            \EndFor
        \EndFor
    \end{algorithmic}
\end{algorithm}

Wird das Entfernen von Überkreuzungen nach diesem Prinzip implementiert, ergeben sich einige Randbedingungen, die es Wert sind erwähnt zu werden.
Aufgrund der geschachtelten Iterationen über die Kanten des Graphs lässt sich eine Zeitkomplexität von $$O(n) = n^2$$ ermitteln, womit der Algorithmus im Rahmen der polynomialen Zeitkomplexitätsklasse liegt. 
Dies bedeutet, dass die Laufzeit des Algorithmus proportional zum Quadrat seiner Eingabemenge wächst.\autocite[15]{Gurski.2010}
Als Eingabemenge können hier Knoten bzw. Kanten eines Graphen behandelt werden, wobei $n$ die Menge der Knoten repräsentiert.
\\
Ein Algorithmus, der Überkreuzungen aus einem Graph entfernt, arbeitet also mit einer ähnlichen Laufzeit wie die Heuristiken, die den Graph vorher erzeugen.
\\\\
Weiterhin ist es möglich, dass durch das Auflösen einer Überkreuzung eine weitere, neue entsteht.
Falls dies so geschieht, dass in den restlichen Iterationen über die Kanten diese Überkreuzung nicht mehr erkannt wird, beispielsweise, wenn dies in der letzten Iteration passiert, dann wird die Überkreuzung nicht vom Algorithmus aufgelöst.
Eine Möglichkeit dies zu umgehen ist durch das rekursive Aufrufen des Algorithmus, damit mehrmals auf Überkreuzungen überprüft wird.
Allerdings besteht hier Bedarf eine solche Implementierung genauer zu untersuchen, was in dieser Arbeit nicht behandelt wird.


\section{Nachbesserung eines Pfads}
\label{sec:after-control}
\subsection{Darstellung des Grundproblems}
Neben den im vorherigen Abschnitt beschriebenen Überkreuzungen kommt es bei den generierten Graphen auch zu solchen, die durch ein einfaches Umlegen der Route verbessert werden können.

\begin{figure}[h]
    \begin{center}
    
    \subfloat[Verbesserungsfähiger Graph \label{subfig:after-control-ex1-1}]{
        \includegraphics[width=0.35\textwidth]{Bilder/afterControl/after_control_ex1.PNG}
    }
    \hfil
    \subfloat[Verbesserter Graph \label{subfig:after-control-ex1-2}]{
        \includegraphics[width=0.35\textwidth]{Bilder/afterControl/after_control_ex2.PNG}
    }
    \end{center}
    \caption{Graph vor und nach der Nachbesserung}
    \label{fig:after-control-ex1}
\end{figure}

Das Beispiel in \vref{fig:after-control-ex1} zeigt einen Graphen mit einer initialen Knotenreihenfolge
$$P_{alt} = k_1,k_2,k_3,k_4$$
Durch das Umlegen zu
$$P_{neu} =k_1,k_3,k_2,k_4$$
erfolgt eine Verringerung der Gesamtdistanz.
Betrachtet man die hier betroffenen Kanten $E_1 = (k_1,k_2)$, $ E_2 = (k_1,k_3)$ und $E_3=(k_2,k_3)$ kann die Distanzverringerung anhand ihrer einzelnen Distanzen und ihres Auftretens in den beiden Graphen erklärt werden.
Da gilt 
$$E_1,E_3\in P_{alt} \textrm{ und } E_2,E_3 \in P_{neu}$$
kann eine Distanzverringerung mit
$$\omega(E_1) + \omega(E_3) < \omega(E_2) + \omega(E_3)$$
erklärt werden.
Im konkreten Beispiel in \vref{fig:after-control-ex1} drückt sich das durch eine Verringerung der Gesamtdistanz um 0,786\ac{LE} aus.


\subsection{Algorithmus zur Nachbesserung}
Ein Algorithmus, der Nachbesserung auf genau diese Art vornehmen soll, kann wie folgt vorgehen.
Betrachte beginnend mit dem zweiten Knoten in einem Pfad jeden Knoten.
Für jeden zu betrachteten Knoten wird jede Kante, an dessen Ende der aktuelle Knoten nicht steht, betrachtet.
Es gilt zu überprüfen, ob es für die Gesamtdistanz besser ist, wenn der aktuelle Knoten in die aktuelle Kante eingefügt wird.
Anders ausgedrückt: $p_i$ mit $i \geq 2,i\in\mathbb{N}$ sei ein Knoten in einem vollständigen Pfad mit $n$ Knoten und $n-1$ Kanten. Zusammen mit $p_i$ wird auch immer eine Kante $e_j=(p_{k-1},p_{k})$ betrachtet, sodass gilt $p_i \not \in e_j$.
Um nun zu überprüfen, ob es möglich ist eine Verbesserung des Graphs vorzunehmen wird überprüft ob 
$$\omega(p_{i-1},p_{i}) + \omega(p_{i},p_{i+1}) + \omega(e_j) > \omega(p_{i-1},p_{i+1}) + \omega(p_{k-1},p_i) + \omega(p_i,p_{k})$$
Ist dies der Fall wird die Reihenfolge der Knoten entsprechend geändert und $p_i$ zwischen $p_{i-1}$ und $p_{i+1}$ entfernt und zwischen $p_{k-1}$ und $p_k$ eingefügt. Wie genau das Einfügen in die Kante funktioniert kann in Algorithmus \vref{alg:after-control-merge} nachvollzogen werden.
\\\\
Eine vollständige Implementierung eines Algorithmus, der Verbesserung an einem bestehenden Graph vornimmt kann wie in Algorithmus \vref{alg:after-control} aussehen.
\begin{algorithm}[h]
    \caption{Nachbesserung eines Pfads}
    \label{alg:after-control}
    \begin{algorithmic}[1]
        \Require Pfad $P$
        \Require $P=p_1,p_2,\cdots,p_n,n>3,n\in \mathbb{N}$
        \For{$a\gets 2$, $a \leq n-1$, $a\gets a+1$}
            \For{$b \gets 3$, $b \leq n$, $b\gets b+1$}
                \If{$a=b$ \textbf{or} $a = b-1$}
                    \State \textsc{continue}
                \EndIf
                \If{$\omega(p_{a-1},p_{a}) + \omega(p_{a},p_{a+1}) + \omega(p_{b-1}, p_b) > \omega(p_{a-1},p_{a+1}) + \omega(p_{b-1},p_a) + \omega(p_a,p_{b})$}
                    \State $P \gets$ \textsc{resolve}($P, (p_{b-1}, p_b), p_a$)
                \EndIf
            \EndFor
        \EndFor\\
        \Return $P$

    \end{algorithmic}
\end{algorithm}
Damit beschreibt sich die Zeitkomplexität dieses Algorithmus mit 
$$f(n) = O(n^2)$$
Womit er, wie der Algorithmus \vref{alg:handle-crossover} Entfernen von Überkreuzungen, quadratisch zur Eingabemenge skaliert.




\chapter{Fallbeispiel Transportation Management SAP}
\section{Implementierung der Algorithmen}
\section{Auswirkungen auf die Routenplanung}


\chapter{Zusammenfassung und Ausblick}
% Performance Insert First

% Gründe für Schwäche

% Ideen zur ausbesserung
%   1. Als ersten weit entfernte Einfügen
%       Performt in manchen Szenarien besser, in manchen schlechter
%       Abhängig von reihenfolge
%   2. Als erstes nahe Knoten einfügen
%       Performt auch besser in manchen Szenarien, aber hat die selben schwächen

% nachträgliche Überarbeitung von existierenden Routen
% Auflösen von Überkreuzungen
% Algorithmus bringt das zu erwartende Ergebnis 
% und Routen ohne Überkreuzungen kürzer

% Nachbesserungs-Algorithmus
% Verringert die Distanz von Routen

% Was lohnt sich
% Hier keine Definition von was sich lohnt aber: Überkreuzungen???

% Da Zwei Variationen nicht wirklich etwas gebracht: Anderes Kriterium?
% Verhalten der Algorithmen bei Größeren mengen von Knoten (Qualität und Laufzeit)


Rückblickend auf die Ergebnisse dieser Arbeit, insbesondere in Bezug auf den Insert-First-Algorithmus lässt sich feststellen, dass eine Heuristik zur Lösung des \ac{TSP} immer Schwächen irgendeiner Form haben wird.
Im Falle des Insert-First-Algorithmus liegt erkennbar in der Reihenfolge, in der die Knoten zu einer vollständigen Route zusammengefügt werden.
Wie gezeigt ist der Insert-First-Algorithmus zwar in der Lage gute, aber ebenso schlechte, Ergebnisse zu generieren.
\\
Ein Versuch diesen Schwächen entgegen zu wirken ist der Insert-Furthest Algorithmus, der die Knoten, die am weitesten voneinander entfernt sind zuerst betrachtet.
Dieser Algorithmus ist in der Lage in einigen Szenarien Ergebnisse einer höheren Qualität zu erzeugen, ebenso leidet er aber unter den gleichen Schwächen wie der Insert-First-Algorithmus und erzeugt durch das späte Einfügen von Knoten, die eine geringe Entfernung zu anderen aufweisen, suboptimale Ergebnisse.
Träg man den Ansatz des Insert-Furthest-Algorithmus in die andere Richtung entspringt dabei der Insert-Closest-Algorithmus, der anstatt der am weitesten entfernten Knoten die nächsten einfügt.
Auch hier lassen sich wieder die gleichen Schwächen, die schon in beiden anderen Variationen des Algorithmus aufgetreten sind, feststellen.
Betrachtet man die Laufzeit der Algorithmen lassen sich keine großen Unterschiede feststellen -- alle drei liegen in der Zeitkomplexitätsklasse $O(n^2)$.
Dies begründet sich darin, dass die Algorithmen basieren auf dem gleichen Prinzip basieren, nur steigt bei Insert-Closest und -Furthest aufgrund des Auswahlkriteriums für Knoten die Laufzeit etwas schneller.
\\\\
Betrachtet man nun also die Algorithmen, die zur nachträglichen Überarbeitung von bereits bestehenden Routen entwickelt wurden, lässt sich über diese sagen, dass sie sich wie erwartet verhalten.
Der Algorithmus zur Auflösung von Überkreuzungen in einer Route löst Überkreuzungen auf und verringert so die Gesamtdistanz einer Route. 
An dieser Stelle sei angemerkt, dass es zu untersuchen gilt, ob das Entfernen von Überkreuzungen für Computer effizienter gestaltet werden kann, beispielsweise durch die Berechnung von Überkreuzungen durch Vektoren, wie in \vref{sec:erkennen-von-ueberkreuzungen} angemerkt.
\\
Bezüglich des Algorithmus zur Nachbesserung von Routen ist, aufbauend auf den in \vref{sec:result} vorgestellten Ergebnissen, eine klare Verbesserung in der Qualität der Ergebnisse zu erkennen.
Die in \vref{sec:result} vorgestellten Algorithmenkombinationen, die den Nachbesserungs-Algorithmus verwenden machen über zwei Drittel der besten Ergebnisse aus und können somit als Lösungen angesehen werden, die häufig sehr gute Ergebnisse erzeugen, die nur geringfügig vom optimalen Ergebnis abweichen.
\\\\
Eine Empfehlung ob oder welche Kombination von Algorithmen am effizientesten arbeitet und in der Praxis verwendet werden sollte, wird hier nicht gegeben, da dieser Arbeit nicht nur Praxisnähe sondern auch eine passende Definition von Effizienz fehlt.
Interessant wäre aber die zukünftige Untersuchung eben dieser Empfehlung.
Weiterhin sei angemerkt, dass die zwei Versuche den Insert-First-Algorithmus zu verbessern nicht zum erwartetenden Ergebnis führen, weshalb sich hier die Untersuchung weiterer Variationen mit anderen Auswahlkriterien empfiehlt.
Außerdem widmet sich diese Arbeit nicht dem Verhalten der vorgestellten Algorithmen bei einer größeren Menge von Knoten.
Auch hier gilt es die Qualität der Ergebnisse und die Laufzeit der Algorithmen zu untersuchen.
\input{Lit/Lit.tex}

%	Anleitungs-Datei anleitung.tex einziehen. Auf diese Weise sollten Sie versuchen, für jedes einzelne
% Kapitel eine eigene Datei anzulegen und mittels input-Kommando einzuziehen.
% \input{anleitung}

%	Literaturverzeichnis
\printbibliography[title=Literaturverzeichnis]
\cleardoublepage

% Der Anhang beginnt hier - jedes Kapitel wird alphabetisch aufgezählt. (Anhang A, B usw.)
\appendix
\ihead{\appendixname~\thechapter} % Neue Header-Definition

% appendix.tex einziehen
\chapter{Anhang}
\section{Bilder}
% TODO: Bilder insertFirst Good (4)
% TODO: Bilder insertFirst Bad (4)
% TODO: Bilder insertFurthest Good (5)
% TODO: Bilder insertFurthest Bad (5?)



% \section{Subtestanhang}

% \chapter{Noch ein Testanhang}

% -------------------------------------------------------------- 
% INSERT CLOSEST BAD COMPLETE
\begin{figure}[h]
    \begin{center}
        \subfloat[$m = 2$\label{app:subfig:insert-closest-BAD-m1}]{%
        \includegraphics[width=0.6\textwidth]{./Bilder/insertClosest/insert_closest_ex_BAD_1.PNG}
        }
    \end{center}
\end{figure}
        % \hfil
\begin{figure}\ContinuedFloat
    \begin{center}
        \subfloat[$m = 3$\label{app:subfig:insert-closest-BAD-m2}]{%
        \includegraphics[width=0.6\textwidth]{./Bilder/insertClosest/insert_closest_ex_BAD_2.PNG}
        }
    \end{center}
\end{figure}
\begin{figure}\ContinuedFloat
    \begin{center}
        \subfloat[$m = 4$\label{app:subfig:insert-closest-BAD-m3}]{%
        \includegraphics[width=0.6\textwidth]{./Bilder/insertClosest/insert_closest_ex_BAD_3.PNG}
        }
    \end{center}
\end{figure}
\begin{figure}\ContinuedFloat
    \begin{center}
        \subfloat[$m = 5$\label{app:subfig:insert-closest-BAD-m4}]{%
        \includegraphics[width=0.6\textwidth]{./Bilder/insertClosest/insert_closest_ex_BAD_4.PNG}
        }
    \end{center}
\end{figure}
\begin{figure}\ContinuedFloat
    \begin{center}
        \subfloat[$m = 5$\label{app:subfig:insert-closest-BAD-m5}]{%
        \includegraphics[width=0.6\textwidth]{./Bilder/insertClosest/insert_closest_ex_BAD_5.PNG}
        }
        \caption{Viele Bilder}
    \end{center}
\end{figure}
\label{app:fig:insert-closest-BAD-complete}




% Ehrenwörtliche Erklärung ewerkl.tex einziehen
\input{ewerkl.tex}


\end{document}
