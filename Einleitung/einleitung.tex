\chapter{Einleitung}
1832 markiert ein ein trauriges Jahr für Literaturliebhaber.
Es ist das Jahr in dem einer der berühmtesten deutschen Schriftsteller stirbt.
Johann Wolfgang Goethes Tod\autocite[110]{Scholte.1951} mag für einige das Ende einer Literaturepoche bedeuten, für die anderen ist es nur ein weiteres unbedeutendes Verschwinden eines überbewerteten Dichters.
Ob und inwiefern Goethes Tod nun von Bedeutung war, soll an dieser Stelle nicht diskutiert werden.
Was sich aber eindeutig sagen lässt, ist, dass trotz noch so großer Verluste unter ihren literarischen Kumpanen, Autoren nie aufhören werden neue Literatur zu publizieren. 
So auch 1832.
\\\\
In Jena erscheint in eben diesem Jahr, wie schon seit längerer Zeit, das \enquote{Ergänzungsblatt zur jenaischen Allgemeinen Literatur-Zeitung}\autocite[232]{o.A..1832}, in dem unter dem Titel \enquote{Kurze Anzeigen}, in gerade einmal 16 Zeilen, ein unscheinbares Buch vorgestellt wird.
Es trägt den Titel \enquote{Der Handlungsreisende wie er sein soll und was er zu thun hat, um Äufträge zu erhalten und eines glücklichen Erfolgs seinen Geschäften gewiß zu sein}.
In ihm beschreibt der Autor auf 206 Seiten den Alltag und die Aufgaben eines Handlungsreisenden und -- viel wichtiger für diese Arbeit -- Reiserouten durch Deutschland und die Schweiz.
Zwar wird hier keine konkrete Problemstellung geschildert, wohl aber das Grundproblem dargelegt: Das möglichst effiziente Anreisen mehrerer Städte.
\\
Was 1832 noch unbedeutend gewirkt haben mag sollte sich Jahre später zu einem der berühmtesten Probleme der Kombinatorik und Optimierung entwickeln.
\\\\
Das Clay Mathematics Institute formuliert im Jahr 2000 sieben Problemstellungen, die das Institut als Jahrtausend Probleme der Mathematik bezeichnet.\autocite{o.A..o.J.}
Unter ihnen befindet sich das \enquote{N versus NP Problem}\autocite{Stephen.o.J.}, welches, ohne groß sich groß in Details zu verlieren, die Frage nach dem Verhältnis der beiden Komplexitätsklassen N und NP stellt.
Um nur einen kurzen Überblick über diese Thematik zu erhalten sei angemerkt, dass es sich bei Problemen der Klasse P um vergleichsweise schnell zu lösende Probleme handelt, während bei NP-Problemen die Lösung, gerade bei steigender Komplexität des Problems, nahezu unerreichbar wird.
Für letztere Problemklasse ist allerdings nicht bewiesen, dass es keine Lösungsverfahren gibt, die diese Probleme schneller lösen, womit sie ein Teil von P werden würden.
\\
Auch wenn diese Probleme erst im jahr 2000 formuliert worden sind, so ist schon die 1832 festgehaltene Aufgabe des Handelsreisenden ein exemplarischen Beispiels dieser Problemklasse und gehört nach heutiger Kenntnisse zu den NP-vollständigen Problemen, was bedeutet, dass eine Lösung nicht nur schwer zu berechnen, sondern zusätzlich auch noch schwer zu beweisen ist.
\\
Trotz der hohen Schwierigkeit dieses Problems und dem Rückgang der klassischen Handlungsreisenden, ist das Problem des Handlungsreisenden (englisch: \acf{TSP}) ein bis heute wichtiges Problem -- gerade bei Unternehmen der Logistikbranche.
Diese stehen vor der Aufgabe die Routen ihrer LKW, Schiffe, Züge, etc. so zu planen, dass zu transportierende Güter ihre Ziele mit einem möglichst geringen Ressourcenaufwand erreichen, sprich, die Strecke, die gefahren werden muss, um alle diese Ziele zu erreichen soll möglichst gering sein.
Da die Berechnung der optimalen Route durch alle gegebenen Ziele auch mit dem besten Computer weltweit nicht in praktikabler Zeit durchführbar ist, werden häufig Heuristiken verwendet, um wenigstens gute Annäherungen an die Lösung schnell generieren zu können.
\\\\
Nachdem das \ac{TSP} zu Beginn beschrieben wird, setzt diese Arbeit genau an diesem Punkt an und diskutiert eine in \vref{sec:insert-first-verfahren} beschriebene Heuristik, analysiert ihre Ergebnisse und Schwächen, sowie die Zeitkomplexität des Algorithmus. 
Aufbauend auf dieser Analyse werden zwei Algorithmen als Variationen der ursprünglichen Heuristik entwickelt, deren Zweck die Ausbesserung von festgestellten Schwächen sein soll.
Auch diese beiden Algorithmen werden diskutiert und auf Basis ihrer Ergebnisse analysiert.
Um die Funktionsweise der Heuristiken nachvollziehen zu können, wird jeder Algorithmus in Pseudocode beschrieben.
\\
Nach der Entwicklung und Analyse der drei Algorithmen werden weiter zwei Algorithmen vorgestellt, deren Aufgabe in der nachträglichen Überarbeitung einer bereits bestehenden Route liegt.
Auch hier werden die Algorithmen wieder in Pseudocode dargestellt, ihre Funktionweise beschrieben und Ergebnisse und Zeitkomplexität analysiert.
\\
Um den Nutzen der ausgearbeiteten Algorithmen ermitteln zu können wird anschließend anhand von Testdaten eine statistische Auswertung verschiedener Kombinationen dieser Algorithmen aufgestellt.
Diese soll die unterschiedlichen Ergebnisse der Algorithmen in Zahlen darstellen und Einblicke über die durchschnittliche Qualität der Routen bringen.
Ob oder welche Kombination von Algorithmen einen praxisbezogenen Nutzen haben könnte und verwendet werden sollte ist nicht Teil dieser Arbeit.
\\\\
Ein Java Programm, welches die in dieser Arbeit beschriebenen Algorithmen implementiert und zur Generierung der in  \vref{sec:result} verwendeten Ergebnisse verwendet wurde findet sich auf GitHub unter \href{https://github.com/Grimmig18/TSP}{https://github.com/Grimmig18/TSP}. 
Die für diese Arbeit verwendete Version liegt unter Commit 2b58a0af38722ae2adca68c6690189e0d6 ac0866.
Das Programm wurde ebenfalls zur Erzeugung der in dieser Arbeit verwendeten Abbildungen verwendet.
\\\\
Für die restliche Arbeit wird die Abkürzung des englischen Namens des Problem des Handlungsreisenden verwendet (\ac{TSP}).
